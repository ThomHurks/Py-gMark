The rise of social network applications over the past years has sparked massive interest in the study of such networks, specifically how to efficiently store and query them using databases. Besides that, many (social) domains are easily and naturally expressible in graphs and networks, such as corporate (financial) networks, criminal networks and even physical network systems such as the internet.

When studying social networks, or more general, graphs, it is obviously necessary to obtain test data. Such test data can be actual social networks or domain graphs, but this data is not always easily accessible. Companies and other institutions, such as governments, may have various reasons why they do not want to publish or share their graph databases. Reasons can vary from financial motivations where a company does not want to give away their core business asset to privacy concerns where data may not even be allowed to be shared. Sometimes the necessary graph data is available, but a researcher may require a much larger dataset to stress test a system, so additional data still needs to be obtained.

The solution to these problems is synthetic graph generation, where graphs with the appropriate size and structural properties are generated. These structural properties may be pre-defined using a graph schema or may be generated as well.
This report examines the state of the art regarding synthetic graph generation, specifically focusing on possible improvements or extensions that can be applied to gMark \cite{Bagan2016GMark:Queries}, a schema driven generator for graphs and associated query workloads.